\documentclass{resume}
\usepackage[left=0.6in,top=0.6in,right=0.6in,bottom=0.6in]{geometry} 
\usepackage{hyperref}
\usepackage{url}

\RequirePackage{color,graphicx}
\usepackage{multicol}
\usepackage{etoolbox}
%Setup hyperref package, and colours for links
\usepackage{hyperref}
\usepackage{fontawesome}
\usepackage{enumitem}
\usepackage{enumerate}
% \usepackage{academicons}
\definecolor{linkcolour}{rgb}{0,0.2,0.6}
\hypersetup{colorlinks,breaklinks,urlcolor=linkcolour, linkcolor=linkcolour}

\usepackage{graphicx}
\newsavebox\CBox
\def\textBF#1{\sbox\CBox{#1}\resizebox{\wd\CBox}{\ht\CBox}{\textbf{#1}}}


\urldef\myurl\url{http://www.cs.cmu.edu/~NatProg/papers/DTSHPS%20paper%207%20-%20one-page-summary-with-references%20v2.pdf}
\urldef\myurl2\url{https://digital.lib.washington.edu/researchworks/bitstream/handle/1773/42857/DTSHPS18-Proceedings-final%20v2.pdf}

\newcommand{\tab}[1]{\hspace{.266\textwidth}\rlap{#1}}
\newcommand{\itab}[1]{\hspace{0em}\rlap{#1}} 
\name{Jane Hsieh} 
\address{\href{https://janeon.github.io}{janeon.github.io } \\  jhsieh2@cs.cmu.edu}
\begin{document}
\begin{rSection}{Education}
{\bf Carnegie Mellon University}{ \href{https://se-phd.isri.cmu.edu/People/students/index.html}{PhD in Software Engineering}\href{https://haiyizhu.com/}{, advised by Haiyi Zhu}} \hfill { August 2020 - Present} \\ 
{\bf Oberlin College} {BA in \href{https://www.cs.oberlin.edu/}{Computer Science (with High Honors)}} and Mathematics \hfill { August 2016 - May 2020} \\
\href{https://www.oberlin.edu/arts-and-sciences/departments/mathematics}
with  \href{https://www.oberlin.edu/cognitive-sciences}{Concentration in Cognitive Science} \hfill {(Major) GPA: (3.77) 3.71} 

% My research focuses on the ways that technological systems and market forces construct socially valuable bodies; I seek to challenge dominant discourses around disability, sexuality, and expertise through community-driven design methods. I hold a BA in Bioethics & Design from UC Berkeley.

{My research is broadly focused on impacts AI-powered systems on online communities and mitigation strategies for algorithmic harms. 
In particular, I explore ways improving the well-being and working conditions of gig workers via technology and policy advancements. 
Leveraging empirical and design techniques, I integrate perspectives of multiple stakeholder groups to develop frameworks for facilitating collectivism and empowerment among gig workers.

}

% Member of {\em IEEE} and {\em American Physical Society} \hfill {Cumulative GPA: 3.71} \\
% IB Diploma recipient \hfill {HS Ranking: 2/22}
\end{rSection}
% \begin{rSection}{Courses}
%  {Artificial Intelligence, Theory of Computation, Computer Architecture, Economics} \hfill {\em Current} \\
%  {Economics and Computation, HCI, Foundations of Analysis, Linguistic Anthropology} \hfill {\em Fall 2018} 
%  {Advanced Algorithms, Systems Programming, Group Theory, Mathematical Modeling} \hfill {\em Spring 2018} 
%  Algorithms, Linear Algebra, Data Structures, Discrete Math, Multivariable Calculus \hfill {\em 2017} \\
%  {Intro to Computer Science, Deconstructing Technology, Calculus II}\hfill {\em Fall 2016}
% \end{rSection}


\begin{rSection}{Publications \& Preprints}
\begin{itemize}
    \item \textBF{Jane Hsieh}, Joselyn Kim, Laura Dabbish, Haiyi Zhu, ``Nip it in the Bud": Moderation Strategies in Open Source Software Projects and the Role of Bots'', \textit{ACM Conference On Computer-Supported Cooperative Work And Social Computing, CSCW ’23}, Minneapolis, MN. \href{https://arxiv.org/pdf/2308.07427.pdf}{arXiv}.
    \item \textBF{Jane Hsieh}, Miranda Karger, Lucas Zagal, Haiyi Zhu, ``Co-Designing Alternatives for the Future of Gig Worker Well-Being: Navigating Multi-Stakeholder Incentives and Preferences", \textit{Designing Interactive Systems Conference, DIS ’23}, Pittsburgh, PA, \href{https://arxiv.org/abs/2302.13436}{arXiv}.
    \item \textBF{Jane Hsieh}, Oluwatobi Adisa, Sachi Bafna, Haiyi Zhu, ``Designing Individualized Policy and Technology Interventions to Improve Gig Work Conditions", {\color{red}Best Paper Award} (1 of 13) from the \textit{Annual Symposium on Human-Computer Interaction for Work 2023, CHIWORK '23}, Oldenburg, DE. \href{https://arxiv.org/abs/2306.12972}{arXiv}.
    \item \textBF{Jane Hsieh}, Yili Hong, Gordon Burtch, Haiyi Zhu, ``A Little Too Personal: Effects of Standardization versus Personalization on Job Acquisition, Work Completion, and Revenue for Online Freelancers", \textit{CHI Conference on Human Factors in Computing Systems, CHI ’22}, New Orleans, LA, \href{https://dl.acm.org/doi/pdf/10.1145/3491102.3517546}{ACM DL}.
    \item Michael Xieyang Liu, \textBF{Jane Hsieh}, Nathan Hahn, Angelina Zhou, Emily Deng, Shaun Burley, Cynthia Taylor, Aniket Kittur, Brad A. Myers, ``Unakite: Scaffolding Developers’ Decision Making About Trade-offs through Capturing and Organizing Web Resources", {\color{red}Best Paper Honorable Mention Award} (top 6 of 93) from the \textit{ACM Symposium on User Interface Software and Technology, UIST'19}, New Orleans, LA, October 20-23, 2019. pp. 67-80. \href{https://dl.acm.org/citation.cfm?id=3347908}{ACM DL} and \href{http://www.cs.cmu.edu/~NatProg/papers/p67-liu-Unakite-UIST.pdf}{local pdf}.
    
    \item Yumi Ijiri, Kathryn L. Krycka, Ian Hunt-Isaak, Hillary Pan, \textBF{Jane Hsieh}, Julie A. Borchers, James J. Rhyne, Samuel D. Oberdick, Ahmed Abdelgawad, Sarah A. Majetich, ``Correlated spin canting in ordered core-shell Fe$_3$O$_4$/Mn$_x$Fe$_{3-x}$O$_4$ nanoparticle polycrystalline assemblies," \textit{Physical Review B} 99(9). March 18, 2019. p. 094421. \href{https://journals.aps.org/prb/abstract/10.1103/PhysRevB.99.094421}{APS DL} and \href{https://janeon.github.io/assets/img/PhysRevB.99.094421.pdf}{local pdf}.

\end{itemize}
\end{rSection}


\begin{rSection}{Lightly-reviewed publications}
\begin{itemize}

    \item Michael Xieyang Liu, Nathan Hahn, Angelina Zhou, Shaun Burley, Emily Deng, \textBF{Jane Hsieh}, Aniket Kittur and Brad A. Myers, ``UNAKITE: Support Developers for Capturing and Persisting Design Rationales When Solving Problems Using Web Resources", \textit{DTSHPS'18 Workshop on Designing Technologies to Support Human Problem Solving} (\href{https://www.cs.washington.edu/dtshps2018/index.html}{DTSHPS'18}) at VL/HCC'2018. Oct. 1, 2018. p. 25. \href{http://www.cs.cmu.edu/~NatProg/papers/DTSHPS%20paper%207%20-%20one-page-summary-with-references%20v2.pdf}{extended abstract} or \href{https://digital.lib.washington.edu/researchworks/bitstream/handle/1773/42857/DTSHPS18-Proceedings-final%20v2.pdf}{full proceedings}.

    \item \textBF{Jane Hsieh}, Michael Xieyang Liu, Brad A. Myers, Aniket Kittur, ``Poster: An Exploratory Study of Web Foraging to Understand and Support Programming Decisions," \textit{2018 IEEE Symposium on Visual Languages and Human-Centric Computing} (VL/HCC'18), October 1 - 4, 2018, Lisbon, Portugal. pp. 305-306. \href{https://ieeexplore.ieee.org/document/8506517}{IEEE DL} and \href{http://www.cs.cmu.edu/~NatProg/papers/p305-hsieh.pdf}{local pdf}.
\end{itemize}
\end{rSection}

\begin{rSection}{Experiences}

\textbf{Data Science Consultant at Upwork Inc.} \hfill  {\em Summer 2021 - Spring 2022} \\
{Supervised by Sibo Lu} \hfill {Remote} 

\textbf{Software Engineer Intern on IBM's Toolbox Team} \hfill  {\em Summer 2020} \\
{\it Developed Slack and Github drivers for IBM's Support Portal} \hfill {Raleigh, NC} 


\textbf{Computer Science Honors Thesis} \hfill  {\em Fall 2019-Spring 2020} \\ {\href{https://digitalcommons.oberlin.edu/cgi/viewcontent.cgi?article=1693&context=honors}{Constructing Effective Stack Overflow Questions}, advised by \href{https://cs.oberlin.edu/~ctaylor/}{Cynthia Taylor}} \hfill {Oberlin, OH}

% {Conducted literature review to find factors of successful questions, and verified with a two-proportions Z-test} \\
% {Developed a dynamic \href{https://github.com/janeon/honors-plugin}{Chrome plugin} that provides actionable suggestions to users constructing questions}

\href{https://www.ibm.com/employment/extremeblue/index.html}{\textbf{Extreme Blue Technical Intern, managed by Ross Grady}} \hfill  {\em Summer 2019} \\
\href{https://github.com/IBM/multicloud-incident-response-navigator}{Open-sourced terminal application for IBM's Multicloud Manager}, \href{https://priorart.ip.com/IPCOM/000262660}{defense patent application} \hfill { Raleigh, NC} 

 
\href{https://www.cmu.edu/scs/isr/reuse/}{\bf REUSE Research Assistant}, advised by \href{https://www.cs.cmu.edu/~bam/}{Brad Myers} \& \href{https://kittur.org/}{Niki Kittur}\hfill {\em Summer 2018-2019} \\
\href{https://unakite.info/}{UNAKITE Tool for Tabulated Decision Making}, mentored by \href{https://lxieyang.github.io/}{Michael Liu} \hfill { Pittsburgh, PA}

{\bf \href{https://www.oberlin.edu/undergraduate-research/programs/strong}{STRONG Research Program}} \hfill {\em 2016 - 2018}\\
Characterizing and Separating Magnetic Nanoparticles, advised by \href{https://www.oberlin.edu/yumi-ijiri}{Yumi Ijiri} \hfill { Oberlin, OH}
\end{rSection}

\begin{rSection}{Teaching}
{\href{https://sites.google.com/andrew.cmu.edu/haii-cmu/}{TA for Human AI Interaction}} \hfill  {\em Fall 2022 and upcoming Fall 2023} 
% {Lectured, graded and held office hours for class of $\approx$40 students} 

{Office hour holder, Grader and Tutor for Algorithms} \hfill {\em Fall 2018 - 2019} 
% { Held office hours, graded, tutored, and held OWLS/recitation sessions.} 

{Lab helper for Python course} \hfill {\em Spring 2017, 2018} 
% { Assisted $\approx$ 20 students debug and find logical errors in weekly Python assignments}

% {\bf Oberlin Workshop \& Learning Sessions (OWLS) Leader for Algorithms} \hfill {\em Fall 2018} \\ { Attended class to plan and lead interactive, non-traditional workshops (weekly)} 

% {\bf Advanced Chinese Drill Session Teacher} \hfill {\em Spring 2017} \\
% { Created lesson plans (after attending class) to lead weekly drills to help students improve speaking fluency}

\end{rSection}

\begin{rSection}{Mentoring}
{Mialy Rasetarinera \& \href{https://echou.notion.site/echou/Erik-Chou-bb52b39b95924d3bbd62a83593e28ab6}{Erik Chou} (HCII REU): developing collective data-exchange portal} \hfill  {\em Summer 2023} 

{\href{https://miranda.karger.org/}{Miranda Karger} \& Lucas Zagal (HCII REU): conducted multi-stakeholder co-design sessions} \hfill  {\em Summer 2022} 

{\href{https://www.joselynkim.com/}{Joselyn Kim}: coded, analyzed, and wrote about open source moderation data} \hfill  {\em Fall - Spring 2021} 

{\href{https://www.oberlin.edu/career/set/soar/soar-leaders}{Sophomore Opportunities \& Academic Resources (SOAR) Leader}} \hfill{\em Fall 2019 - Spring 2020}

\end{rSection}

\begin{rSection}{HONORS/AWARDS}
{National Science Foundation Graduate Research Fellowship} \hfill{2022} 

{2020 Annual R.J. Thomas Award for an Outstanding Computer Science Student} \hfill{2020} 

{Clare Boothe Luce Scholarship at Oberlin College} \hfill{\em Tuition scholarship for Fall 2018 - Spring 2019} 

% {\bf John F. Oberlin Scholarship}\hfill{\em 2016 - 2020} 

{STRONG Scholarship \& IB Diploma recipient }\hfill{\em Summer 2016}

\end{rSection}

\begin{rSection}{Service \& Volunteering} 
{Subcommittee Chair Assistant to \href{https://chi2023.acm.org/subcommittees/selecting-a-subcommittee/}{Interaction Beyond the Individual Subcommittee}} \hfill{\em CHI 2023} 
% Tech support and assistance to subcommittee chairs during and prior to PC meetings 

{Member of the \href{https://scs-phd-deans-committee.github.io/working-groups}{DPAC Undergrad Research Engagement Working Group}} \hfill{\em Fall 2021 - ongoing} 
% Plan and organize research mixers and panels

{\href{https://www.oc2020.oberlincollegelibrary.org/}{Web Dev for Digital Yearbook}} \hfill{\em Summer 2020} 
% Designed and implemented visual layout to the digital yearbook using Omeka Classic, CSS, HTML and PHP.


{\href{https://advance.oberlin.edu/events/2020/10/13/uncovering-covid-19-critical-liberal-arts-perspectives-lecture-series}{Uncovering Covid Workshop Leader}} \hfill{\em Spring 2020}
% Planned, trained for and led weekly discussions for 15 admitted Oberlin students on a half-module course exploring Covid-19 from a variety of disciplines. Attended weekly lectures by professors from 8 departments.

{REUSE (Software Engineering REU at CMU) Admissions Panel}\hfill{\em 2021-2023}	

{Reviewer for ACM CHI} \hfill {\em 2020, 2023}

{Student Volunteer for ACM CHI and DIS} \hfill {\em 2023}
% \begin{rSection}{Reviewing}

% \end{rSection}

% \href{http://www.cs.oberlin.edu/~csmc/officers.php}{\textbf{Computer Science Majors Committee Member}} \hfill {\em Fall 2018 - Spring 2020} \\
% { Organized department activities, updated committee websites, held weekly office hours} 

% {\bf ACM ICPC East Central NA Regional Contest}\hfill{\em Fall 2017} \\
% {Received Honorary Mention} 

% \textbf{Spoken languages: }{Mandarin, Shanghainese, Spanish} 

% {\bf Other interests: }{Violin, running, rock climbing, baking, reading}
% \end{rSection}

% % \begin{rSection} {Other Projects}
% % {\bf Automated Lab Helper (AI project)
% % }\hfill  {\em Spring 2019} \\
% % Created program that lints code, sorts errors and recommends solutions for beginning CS students at Oberlin

% % {\bf Frontend Dev for Conceptum: a Question Repository for Educators}\hfill  {\em Winter - Spring 2019} \\
% % Implemented Angular interface components for an iterative question development site designed for professors

% % {\bf Taskat (HCI project)
% % }\hfill  {\em Fall 2019} \\
% % Designed and implemented {React} Electron desktop app to help users to record, and track time of tasks

% % {\bf Star and Galaxy Clustering (Systems project)
% % }\hfill {\em Spring 2018} \\
% % {Implemented K-means in {$C^{++}$}, used SIMBAD catalogue to query $\sim$ 1000 stars and gnuplot as frontend}

% % \textbf{Food Optimization and Peer Tutoring Messaging apps}\hfill {\em Fall 2018 \& Winter 2017} \\
% % { Developed prototype iOS apps using Swift 2 \& 3} \hfill { PennApps \& Oberlin}
% % \end{rSection}

% \begin{rSection}{Awards \& Honors} 
% {\bf 2020 R.J. Thomas Award for Outstanding Computer Science Student} \hfill{\$500} \\
% Awarded per annum to one senior in the Computer Science Department

% {\bf Clare Boothe Luce Scholarship at Oberlin College} \hfill{\$38,808} \\
% Awarded per annum to a woman studying in a scientific field

% {\bf 2018 Computing Research Association for Women GHC Research Scholarship}\hfill{\$500} 

% {\bf John F. Oberlin Scholarship } \hfill{\$69,000}

% {\bf Oberlin College Grant } \hfill {\$23,538}

% {\bf Oberlin ASG Endowed Scholarship} \hfill {\$20,324}

% {\bf STRONG Scholar}\hfill{\$2,500} \\
%  Researched in 2016 and mentored 2017's cohort of students from underrepresented backgrounds
% \end{rSection}

% \begin{rSection} {Conferences \& Workshops Attended}

% {\bf Symposium on Visual Language and Human-Centric Computing} \hfill {October 2018} \\
% Presented poster and short talk. Submitted extended abstract and workshop paper \hfill {\em Lisbon, Portugal}

% {\bf Grace Hopper Celebration} \hfill {Fall 2018} \\
% Research scholar with the Computing Research Association for Women \hfill {\em Houston, Texas}

% {\bf Ohio Summer Research Symposium} \hfill {July 2017}\\
% Gave talk on modeling PASANS data of Manganese Ferrite Nanoparticles \hfill {\em Ohio Wesleyan University}

% {\bf Celebration of Undergraduate Research} \hfill {Oberlin, OH} \\
%  {- Poster: An Exploratory Study of Web Foraging to Understand \& Support Programming Decisions}\hfill {\em 2018} 
 
%  {- Poster: Determining the Magnetic Structure of Ferrite Nanoparticles} \hfill {\em 2017} 
 
%  {- Poster: Improving the Design of a Magnetic Nanoparticle Separation Channel} \hfill {\em 2016}
\end{rSection}
\end{document}
